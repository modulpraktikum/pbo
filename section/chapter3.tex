\chapter{Class Pada Phyton}
\par Class merupakan sebuah objek yang di dalam nya biasanya terdapat beberapa metode yang memang merupakan isi dari sebuah class ini. Class dan metode ini biasa di sembut sebagai OOP atau object oriented programing. Dan OOP ini memang fungsinya untuk memudahkan proses atau kegiatan programing kita. Class ini merupakan sebuah objek yang lebih complex dengan di dalamnya berisi beberapa metode.
\section{Cara Membuat Sebuah Class Pada Python}
Untuk membuat sebuah class ini, harus kita awali dengan sebuah kata kunci. Yaitu “class” yang kemudian di ikuti dengan “nama class nya”. Dan yang terakhir adalah tanda kurung buka dan tutup serta tanda titik dua “()” dan ‘:’.  untuk lebih mudahnya kita bisa lihat atau simak contohnya di bawah ini:
\lstinputlisting[language=Python, breaklines=true, caption=koding]{src/25.py}
\newpage \section{Cara memanggil sebuah class dan metode didalamnya.}
Untuk memanggil sebuah class, sama saja seperti layak nya memanggil metode.. Kita cukup menyebutkan nama classnya dengan di akhiri dengan tanda kurung buka dan tutup seperti di bawah ini.
\lstinputlisting[language=Python, breaklines=true, caption=koding]{src/26.py}
Untuk memanggil metodenya, kita cukup menggunakan memanggil class yang kemudian di ikuti dengan pemanggilan nama metode yang tersedia di dalam class tersebut dengan di pisahkan oleh tanda titik. Seperti di bawah ini.
\lstinputlisting[language=Python, breaklines=true, caption=koding]{src/27.py}
Untuk memudahkan pemanggilan metode ini, kita bisa menampung class nya ke dalam sebuah variabel terlebih dahulu. Yang kemudian kita panggil metodenya seperti di bawah ini.
\lstinputlisting[language=Python, breaklines=true, caption=koding]{src/28.py}
Di dalam sebuah class,biasanya terdapat sebuah metode yang namanya sudah di sediakan oleh python. Namanya adalah “\_\_init\_\_”. Dan jika contoh di atas kita tambahkan \_\_init\_\_ maka kurang lebih akan seperti berikut ini.
\lstinputlisting[language=Python, breaklines=true, caption=koding]{src/29.py}
Dan sama seperti metode, kita bisa menggunakan atau mengirim sebuah nilai di dalamnya atau tidak. Untuk mengirimnya sama saja. Kita cukup memasukkan sebuah variabel di dalam tanda Kurung pada metode \_\_init\_\_. Dan ingat, bukan pada tanda kurung milik clannya ya. Seperti dibawah ini:
\lstinputlisting[language=Python, breaklines=true, caption=koding]{src/30.py}
Untuk memanggil sebuah class yang memiliki parameter, tentu kita harus memasukkan sebuah nilai saat pemanggilannya. Seperti yang ada di bawah ini.
\lstinputlisting[language=Python, breaklines=true, caption=koding]{src/31.py}
Penjelasan mengenai \_\_init\_\_, metode ini merupakan metode yang akan langsung dijalankan ketika class kita di panggil nantinya. Jadi kita tidak perlu memanggil metodenya secara manual seperti metode - metode yang lain seperti yang sudah saya jelaskan di atas.
\section{Contoh dan pemanfaatan sebuah class pada python.}
Setelah kita mengetahui cara membuat dan cara memanggilnya, maka sekarang saya akan mencoba untuk melihat contoh dan pemanfaatan dari sebuah class ini. Hal ini tentu agar membuat lebih paham mengenai apa yang dimaksud dengan class pada python ini. Contoh programnya di seperti bawah ini:
\lstinputlisting[language=Python, breaklines=true, caption=koding]{src/32.py}
