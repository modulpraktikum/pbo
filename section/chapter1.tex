\chapter{Python}
Python merupakan bahasa pemrograman tingkat tinggi yang diracik oleh Guido van Rossum.

Python banyak digunakan untuk membuat berbagai macam program, seperti: program CLI, Program GUI (desktop), Aplikasi Mobile, Web, IoT, Game, Program untuk Hacking, dsb.

Python juga dikenal dengan bahasa pemrograman yang mudah dipelajari, karena struktur sintaknya rapi dan mudah dipahami.
 \section{Hello World Python}
 \subsection{Teori}
 Python memang sangat sederhana dibandingkan bahasa yang lainnya. Tidak perlu ini dan itu untuk membuat program Hello World!.
Bahkan tagline di websitenya menjelaskan, kalau python akan membuatmu bekerja lebih cepat dan efektif.
Berikut salah satu contoh dasar perbedaan python dengan bahasa pemograman lain:
\par\textbf{C++ "Hello World"}
\begin{lstlisting}
#include <iostream.h>
main()
{
count << "hello world!";
}
retun 0
\end{lstlisting}
\par \textbf{Java "Hello World"}
\begin{lstlisting}
class HelloWorldApp
{
public static void main(String[] args)
    {
    System.out.printin("Hello World!");
    }
}
\end{lstlisting}
\par \textbf{Python}
\begin{lstlisting}
print "Hello World"
\end{lstlisting}

